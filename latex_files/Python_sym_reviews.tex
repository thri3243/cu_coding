\documentclass[12pt]{article}

\usepackage[top=1in, bottom=1in, left=1in, right=1in]{geometry}
\usepackage{amsmath}
\usepackage{amssymb}
\usepackage{graphicx}
\usepackage{color}

\title{LaTex Crash Course}
\author{Tad Riley}
\date{November 6, 2017}

\begin{document}
\newpage

\section{Symbol Review}
\subsection{Keywords}%----------------------------------------------------
\begin{tabular}{|l|p{3in}|l|}
	\hline
	\textbf{Keyword} & \textbf{Description} & \textbf{Example}\\ \hline
	and & Logical \textit{and} & True and False == False\\
	as & Part of the \textit{with-as} statement. & with X as Y: Pass\\
	assert & Assert (ensure) that something is true. & assert False, "Error!"\\
	break & Stop this loop right now. & while True: break\\
	class & Define a class & class Person(object)\\
	continue & Don't process more of the loop, do it again. & while True: continue\\
	def & Define a function & def X (): pass\\
	del & Delete from dictionary. & del X[Y]\\
	elif & Else if condition. & if: X; elif: Y; else: J\\
	else & Else condition. & if: X; elif: Y; else: J\\
	except & If an exception happens, do this. & except ValueError, e: print(e)\\
	exec & Run a string as python. & exec print("hello")'\\
	finally & Exceptions or not, finally do this no matter what. & finally pass\\
	for & Loop pver a collection of things. & for X in Y: pass\\
	from & Importing specific parts of a module. & fron X import Y\\
	global & Declare that you want a global variable. & global X\\
	if & If condition. & if: X; elif: Y; else: J\\
	import & Import a module into this one to use. & import os\\
	in & Part of \textit{for-loops}. Also a test of X in Y. & for X in Y: pass also 1 in [1] == True\\
	is & Like == to test wquality. & 1 is 1 == True\\
	lambda & Create a short anonymous fuction. & s = lambda y: y ** y; s(3)\\
	not & Logical not. & not Truw == False\\
	or & Logical or. & True or False == True\\
	pass & This block is empty. & def empty(): pass\\
	print & Print thi string. & print('this string')\\
	raise & Raise an exception when things go wrong. & raise ValueError("No")\\
	return & Exit the function with a return value. & def X(): return Y\\
	try & Try this bock, and if exception, go to except. & try: pass\\
	while & While loop. & while X: pass\\
	with & With an expression as a variable do. & with X as Y: pass\\
	yield & Pause here and return to caller. & def X() yield; X().next\\

	\hline
\end{tabular}

\subsection{Data Types}
\begin{tabular}{|l|l|p{3in}|}
	\hline
	\textbf{Type} & \textbf{Description} & \textbf{Example}\\ \hline
	True & True boolean value. & True or False == True\\
	False & False boolean value. & False and True == False\\
	None & Represents "nothing" or "no value". & x = None\\
	bytes & Stores bytes, maybr of text, PBG, file, etc. & x = b"hello"\\
	strings & Stores textual information. & x = "hello"\\
	numbers & Stores integers. & i = 100\\
	floats & Stores decimals. & i = 10.389\\
	lists & Stores a list of things. & j = [1,2,3,4]\\
	dicts & Stores a key=value mapping of things. & e = {'x': 1, 'y': 2}\\
	
	\hline
\end{tabular}

\subsection{String Escape Sequences}
\begin{tabular}{|l|p{1.5in}|}
	\hline
	\textbf{Escape} & \textbf{Description}\\ \hline
	\textbackslash\textbackslash & Backslash\\
	\textbackslash ' & Single-quote\\
	\textbackslash " & Double-quote\\
	\textbackslash a & Bell\\
	\textbackslash b & Backspace\\
	\textbackslash f & Formfeed\\
	\textbackslash n & Newline\\
	\textbackslash r & Carriage\\
	\textbackslash t & Tab\\
	\textbackslash v & Vertical tab\\
	
	\hline
\end{tabular}



\subsection{Old Style String Formats}
\begin{tabular}{|l|l|l|}
	\hline
	\textbf{Escape} & \textbf{Description} & \textbf{Example}\\ \hline
	\%d & Docimal integers (not floating point). & "\%d" \% 45 == '45\\
	\%i & Same as \%d. & "\%i \% 45 == '45'\\
	\%o & Octal number. & "\%o" \% 1000 == '1750'\\
	\%u & Unsigned decimal. & "\%u" \% -1000 == '-1000'\\
	\%x & Hexadecimal lowercase. & "\%x" 1000 == 3e8\\
	\%X & Hexadecimal uppercase. & "\%X" 1000 == 3E8\\
	\%e & Exponential notation, lowercase 'e'. & "\%e" \% 1000 == '1.000000e+03'\\
	\%E & Exponential notation, uppercase 'E'. & "\%E" \% '1000 == 1.00000E+03'\\
	\%f & Flointing point real number. & "\%f" \% 10.34 == '10.340000'\\
	\%F & same as \%f. & "\%F" \% 10.34 == '10.3400'\\
	\%g & Either \%f or \%e, whicever is shorter. & "\%g" \% 10.34 == '10.34'\\
	\%G & Same as \%g but uppercase. & "\%G" \% 10.34 == '10.34'\\
	\%c & Character format. & "\%c" \% 34 == '"'\\
	\%r & Repr format (debugging format). & "\%r" \% int == "\textless type 'int'\textgreater "\\
	\%s & String format. & "\%s there" \% 'hi' == 'hi there'\\
	\%\% & A percent sign. & "\%g\%\%" \% 10.34 == '10.34\%'\\
	
	
	\hline
\end{tabular}

\subsection{Operators}
\begin{tabular}{|l|l|l|}
	\hline
	\textbf{Operator} & \textbf{Description} & \textbf{Example}\\ \hline
	+ & Addition & 2 + 4 ==6\\
	- & Subtraction & 2 - 4 == -2\\
	* & Multiplication & 2 * 4 == 8\\
	** & Power of & 2 ** 4 == 16\\
	/ & Division & 2 / 4  == 0.5\\
	// & Floor division & 2 // 4 == 0\\
	\% & String interpolate or modulus & 2 \% 4 == 2\\
	\textless & Less than & 4 \textless ~4 == False\\
	\textgreater & Greater than & 4 \textgreater ~4 == False\\
	\textless = & Less than equal & 4 \textless = 4 == True\\
	\textgreater = & Greater than equal & 4 \textgreater = 4 == True\\
	== & Equal & 4 == 5 == False\\
	!= & Not equal & 4 != 5 == True\\
	( ) & Parenthesis & len('hi') == 2\\
	$[~~]$ & List brackets & [1,3,4]\\
	\{ ~\} & Dict curly braces & \{'x': 5, 'y': 10\}\\
	@ & At (decorators) & @classmethod\\
	, & comma & range(0,10)\\
	: & Colon & def X():\\
	. & Dot & self.x = 10\\
	= & Assign equal & x = 10\\
	; & Semi-colon & print("hi"); print("there")\\
	+= & Add and assign & x = 1; x += 2\\
	-= & Subtract and assign & x = 1; x -= 2\\
	*= & Multiply and Assign & x = 1; x *= 2\\
	/= & Divide and assign & x = 1; x /= 2\\
	//= & Floor divide and assign & x = 1; x //= 2\\
	\%= & Modulus assign & x = 1; x \%= 2\\
	**= & Power assign & x = 1; x **= 2\\
	
	
	\hline
\end{tabular}







\end{document}